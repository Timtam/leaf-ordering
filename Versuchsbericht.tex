\documentclass[a4paper, 10pt, twoside, onecolumn, parskip]{scrartcl}
%\KOMAoptions{DIV=last}
\usepackage[left=4cm,right=2cm,top=2cm,bottom=4cm]{geometry}
\usepackage[ngerman]{babel}    % deutsche Einstellungen
\usepackage[utf8]{luainputenc}    % Eingabekodierung für Umlaute im Quellcode
\usepackage[T1]{fontenc}
\PassOptionsToPackage{hyphens}{url}
\usepackage[draft=false, unicode, breaklinks, ngerman, pdfdisplaydoctitle, pdfpagelayout=SinglePage%, colorlinks
]{hyperref}
\usepackage[ocgcolorlinks]{ocgx2}
\usepackage{amssymb} % mathematische Sonderzeichen
\usepackage{amsmath}
\usepackage{enumitem}
\usepackage{float}
\usepackage{slashbox}
\usepackage{comment}
\usepackage{siunitx}
\usepackage[autostyle=true,german=quotes]{csquotes}
\usepackage[backend=biber, style=alphabetic, block=ragged, backref]{biblatex}
\addbibresource{references.bib}

%\renewcommand*{\fps@figure}{htbp}

\setlist{itemsep=.5em, parsep=.0em}
%\hyphenchar\font=\string"7F
\raggedbottom

\title{Übung Algorithm Engineering} % Titel
\subtitle{Forschungsbericht}
\author{{Toni Barth} und {Max Haarbach}} % Autor
\date{\today}            % \today wird durch das aktuelle Datum ersetzt

\begin{document}
    \maketitle                % hier wird der Titel dann gedruckt
    %\tableofcontents

    \section{Heuristiken} \label{sec:heuristiken}

    \subsection{Heuristik 1: Zufallsdrehungen} \label{subsec:heuristic1}

    Die erste Heuristik führt eine bestimmte Anzahl an Drehungen, die von der Größe der Instanz abhängt, an zufällig ausgewählte Knoten aus.
    Dieser Vorgang wird wiederum je nach Größe der Instanz mehrfach durchgeführt und am Ende die Sortierung mit dem geringsten Abstand als Ergebnis ausgegeben.

    \subsection{Heuristik 2: Optimal Leaf Ordering} \label{subsec:heuristic2}

    Die zweite Heuristik nutzt das Verfahren, das Bar-Joseph und weitere für eine möglichst schnelle und optimale Sortierung von hierarchisch geclusterten Datensätzen entwickelt haben~\cite{bar2001fast}.
    Dabei wird folgender rekursiver Ansatz verfolgt:
    Sollen Kosten für einen bestimmten Knoten berechnet werden, setzen sich diese aus den Kosten der beiden Kindknoten und dem Abstand der beiden inneren Blätter dieser beiden Teilbäume.
    Sofern der Knoten, für denen Kosten berechnet werden sollen, ein Blatt bzw. einen Datensatz darstellt, betragen dessen Kosten $0$.
    Dies ist daher das Rekursionsende.
    Begonnen wird üblicherweise mit dem Wurzelknoten, da man dadurch am Ende auch die gesamten Abstandskosten berechnet hat.

    \section{Ziele} \label{sec:ziele}

    Durch die Experimente sollen sowohl
    \begin{itemize}
        \item die Laufzeiten der Heuristiken bei unterschiedlichen Größenordnungen bezüglich der Anzahl der Testobjekte als auch
        \item die Güte aufgrund der Ähnlichkeiten zu den jeweiligen Originalbildern
    \end{itemize}
    ermittelt und verglichen werden.

    \section{Faktoren} \label{sec:faktoren}

    Beim \enquote{Leaf-ordering} sind lediglich 2 Faktoren von Bedeutung:

    Zum Einen bestimmt die Größe der Bilder, die im Endeffekt die Anzahl der Testobjekte widerspiegelt, die Laufzeit der Heuristiken.
    Zum Anderen spielt auch deren Struktur oder Art eine Rolle, die sich allerdings schwer in konkrete Messgrößen oder Werte fassen lassen.

    \section{Testinstanzen} \label{sec:testinstanzen}

    Gemäß der~\nameref{sec:faktoren} werden auch die Testinstanzen, die durch Grauwert-Bilder realisiert sind, in die entsprechenden Kategorien unterteilt:

    \begin{itemize}
        \item Größen:
        \begin{itemize}
            \item 10
            \item 50
            \item 100
            \item 500
            \item 1000
            \item 2500
            \item 5000
        \end{itemize}
        \item Arten:
        \begin{itemize}
            \item (symmetrische) Testbilder
            \item Fotos der realen Welt
            \item Farb- bzw. Grau-Übergänge
        \end{itemize}
    \end{itemize}

    \section{Testergebnisse}

    Es wurden erst einmal nur Tests für die Instanzen der Größen 10, 50 und 100 durchgeführt, da ab einer Größe von 500 die Laufzeit der zweiten Heuristik stark zunimmt.
    Dabei wurden insgesamt 3 Durchläufe durchgeführt, in denen jeweils eine Testinstanz mit bestimmter Art und Größe durch die beiden Heuristiken sortiert wurde.
    Diese Tests wurden auf dem Linux-Server "Anubis" der MLU ausgeführt.

    \subsection{Ergebnis-Qualität} \label{sec:qualität}

    Als Maß für die Qualität der Ergebnisse wird die Summe der Abstände aller benachbarten Blattpaare genutzt, sodass bei 10 Blättern 9 Abstände zu addieren sind.
    Die Abstände wiederum werden durch den euklidischen Abstand der entsprechenden Spaltenvektoren berechnet.
    Für die gemischten und sortierten Instanzen sind die Werte im~\autoref{sec:app_qualität} zu finden, wobei von letzteren am Ende der Durchschnitt gebildet wird.

    Dabei fällt auf, dass die Werte der zweiten Heuristik nicht in allen Fällen gleich sind, wobei der Algorithmus aber immer die beste Lösung finden sollte.
    Dessen Ursache kann zum Beispiel in Fehlern der Implementation liegen.

    %\subsection{Abstandssummen der Original-Instanzen} \label{subsec:sum_original}
    %
    %\begin{center}
    %    \begin{tabular}{|r|r|r|r|r|}
    %        \hline
    %        \backslashbox{Art}{Größe} & 10 & 50 & 100 & 500 \\\hline
    %        g1 & & & &  \\\hline
    %        g2 & & & &  \\\hline
    %        p1 & & & &  \\\hline
    %        p2 & & & &  \\\hline
    %        t1 & & & &  \\\hline
    %        t2 & & & &  \\\hline
    %    \end{tabular}
    %    \captionof{table}{Abstandssummen der originalen Testinstanzen}
    %\end{center}

    %\subsection{Abstandssummen der gemischten Bilder} \label{subsec:sum_shuffle}

    %\subsection{Abstandsummen der sortierten Bilder} \label{subsec:sum_sort}

    \subsection{Laufzeiten der Heuristiken} \label{sec:laufzeiten}

    Auch bei den Laufzeiten wurden je Art und Größe der Instanz 3 Messungen durchgeführt, von denen am Ende der Durchschnitt berechnet wird.
    Die Einheiten der Messungen sind jeweils Sekunden (s).
    Die Tabellen der Laufzeitenmessungen sind im~\autoref{sec:app_laufzeiten} aufgeführt.

    \section{Wilcoxon Signed-Rank Test}

    Als Vergleich der beiden Heuristiken wird ein Wilcoxon Signed-Rank Test auf Basis der Durchschnittswerte der Ergebnis-Qualitäten und Laufzeiten der beiden Heuristiken über alle Größen und Arten durchgeführt.
    Daraus resultieren 18 Datenpunkte pro Heuristik für den Test.

    Anhand der Tabelle der kritischen Werte für den Wilcoxon-Test (\cite{wilcoxon_table}) kann für diese Stichprobengröße und einem Signifikanzniveau von $\alpha{} = 0,05$ ein kritischer Wert von $40$ abgelesen werden.
    Liegt der aus dem Test ermittelte Wert darunter, befindet er sich im kritischen Bereich und die Nullhypothese, dass die Messwerte ähnlich verteilt sind, kann abgelehnt werden.

    Aus Tabelle 1 ist erkennbar, dass ausschließlich positive Ränge vorhanden sind, was auf die zweite Heuristik mit dem Algorithmus der besten Lösung zurückzuführen ist.
    Daraus resultiert die kleinere Summe der negativen Ränge mit $0$ und liegt damit unter dem kritischen Wert.

    \begin{figure}[H]
        \begin{center}
            \begin{tabular}{|r|l|S[table-format=5.7]|S[table-format=5.7]|S[table-format=4.9]|c|S[table-format=3]|S[table-format=3]|}
                \hline
                Größe & Art & {Heuristik 1} & {Heuristik 2} & {Differenz} & Vorz. &     \multicolumn{2}{|c|}{Rang} \\
                 & & & & & & {pos.} & {neg.} \\\hline
                 10 & g1 & 566,6235268 & 426,75209   & 139,8714368 & + & 7   & \\
                 10 & g2 & 308,6377425 & 249,8738006 & 58,76394195 & + & 5   & \\
                 10 & p1 & 545,985978  & 522,4447331 & 23,54124498 & + & 4   & \\
                 10 & p2 & 396,6925964 & 373,4007865 & 23,29180987 & + & 3   & \\
                 10 & t1 & 198,4223745 & 192,9098592 & 5,512515253 & + & 1   & \\
                 10 & t2 & 848,9135363 & 838,2630473 & 10,65048897 & + & 2   & \\\hline
                 50 & g1 & 2725,438841 & 1295,841883 & 1429,596958 & + & 15  & \\
                 50 & g2 & 1475,971813 & 604,6365335 & 871,3352796 & + & 12  & \\
                 50 & p1 & 6908,250429 & 6587,512858 & 320,7375704 & + & 8   & \\\
                 50 & p2 & 6026,672219 & 5492,529012 & 534,1432074 & + & 9   & \\\
                 50 & t1 & 2935,220663 & 2798,969456 & 136,2512073 & + & 6   & \\\
                 50 & t2 & 10971,5055  & 10183,3819  & 788,1236028 & + & 11  & \\\hline
                100 & g1 & 4677,81622  & 1902,730879 & 2775,085342 & + & 18  & \\
                100 & g2 & 2286,828587 & 1072,16236  & 1214,666226 & + & 14  & \\
                100 & p1 & 19980,72542 & 18371,62428 & 1609,101134 & + & 16  & \\
                100 & p2 & 16975,622   & 14399,11553 & 2576,506477 & + & 17  & \\
                100 & t1 & 10733,01992 & 9955,006258 & 778,0136641 & + & 10  & \\
                100 & t2 & 19988,81125 & 18823,77994 & 1165,031308 & + & 13  & \\\hline
                $\sum{}$ & &           &             &             &   & 171 & 0 \\\hline
            \end{tabular}
            \captionof{table}{Wilcoxon Signed-Rank Test für Ergebnis-Qualität}
        \end{center}
    \end{figure}

    Auch bei den Laufzeiten zeigt sich nach Tabelle 2 die kleinere Summe der positiven Ränge mit $21$.
    Dadurch liegt auch diese unter dem kritischen Wert.

    \begin{figure}[H]
        \begin{center}
            \begin{tabular}{|r|l|S[table-format=1.4]|S[table-format=2.4]|S[table-format=2.4]|c|S[table-format=2.1]|S[table-format=3]|}
                \hline
                Größe & Art & {Heuristik 1} & {Heuristik 2} & {Differenz} & Vorz. &     \multicolumn{2}{|c|}{Rang} \\
                 & & & & & & {pos.} & {neg.} \\\hline
                 10 & g1 & 0,002  & 0,001   & 0,001   & - & 3,5 & \\
                 10 & g2 & 0,002  & 0,001   & 0,001   & - & 3,5 & \\
                 10 & p1 & 0,002  & 0,001   & 0,001   & - & 3,5 & \\
                 10 & p2 & 0,002  & 0,001   & 0,001   & - & 3,5 & \\
                 10 & t1 & 0,002  & 0,001   & 0,001   & - & 3,5 & \\
                 10 & t2 & 0,002  & 0,001   & 0,001   & - & 3,5 & \\\hline
                 50 & g1 & 0,0427 & 0,3283  & 0,2856  & + &     & 10 \\
                 50 & g2 & 0,042  & 0,6557  & 0,6137  & + &     & 12 \\
                 50 & p1 & 0,043  & 0,137   & 0,094   & + &     & 8  \\\
                 50 & p2 & 0,043  & 0,126   & 0,083   & + &     & 7  \\\
                 50 & t1 & 0,0423 & 0,381   & 0,3387  & + &     & 11 \\\
                 50 & t2 & 0,043  & 0,1667  & 0,1237  & + &     & 9  \\\hline
                100 & g1 & 0,1757 & 11,404  & 11,2283 & + &     & 15 \\
                100 & g2 & 0,1747 & 14,6813 & 14,5066 & + &     & 17 \\
                100 & p1 & 0,1793 & 5,3723  & 5,193   & + &     & 13 \\
                100 & p2 & 0,176  & 22,9617 & 22,7857 & + &     & 18 \\
                100 & t1 & 0,1753 & 7,7143  & 7,539   & + &     & 14 \\
                100 & t2 & 0,1767 & 12,1773 & 12,0006 & + &     & 16 \\\hline
                $\sum{}$ & &      &         &         &   & 21  & 150 \\\hline
            \end{tabular}
            \label{t:wilcoxon_runtimes}
            \captionof{table}{Wilcoxon Signed-Rank Test für Laufzeiten}
        \end{center}
    \end{figure}

    Somit agieren die beiden Heuristiken sowohl hinsichtlich der Qualität als auch der Laufzeit sehr unterschiedlich, zumindest in den Größenordnungen von 10 bis 100 Elementen der zu sortierenden Daten.

    Des Weiteren ist aus den Werten des Wilcoxon-Test aber auch die eindeutige Tendenz zu einer besseren Laufzeit der ersten Heuristik ($21$ im Vergleich zu $150$) sowie zu einem besseren Ergebnis der zweiten Heuristik ($0$ im Vergleich zu $171$) erkennbar.

    \newpage
    \appendix
    \section{Ergebnis-Qualität} \label{sec:app_qualität}

\subsection{Abstandssummen der gemischten Bilder} \label{subsec:app_sumShuffle}

%\begin{figure}[H]
%    \begin{minipage}{.45\textwidth}
\begin{center}
    \begin{tabular}{|r|S[table-format=3.9]|S[table-format=3.9]|S[table-format=3.9]|}
        \hline
        \backslashbox{Art}{Messlauf} & {\#1} & {\#2} & {\#3} \\\hline
        g1 & 834,302888687 & 763,215338135 & 915,014689799 \\\hline
        g2 & 523,503612176 & 523,503612176 & 573,212169601 \\\hline
        p1 & 715,996547763 & 715,996547763 & 713,029264826 \\\hline
        p2 & 428,438632521 & 388,942148901 & 406,337541942 \\\hline
        t1 & 248,472480733 & 263,107387688 & 250,725344778 \\\hline
        t2 & 974,582228968 & 885,182192758 & 950,296186762 \\\hline
    \end{tabular}
    \captionof{table}{Abstandssummen der gemischten Instanzen der Größe 10}
\end{center}
%    \end{minipage}\hfill%
%    \begin{minipage}{.45\textwidth}
\begin{center}
    \begin{tabular}{|r|S[table-format=5.8]|S[table-format=5.8]|S[table-format=5.8]|}
        \hline
        \backslashbox{Art}{Messlauf} & {\#1} & {\#2} & {\#3} \\\hline
        g1 & 3386,10358879 & 2756,28773204 & 3230,96412561 \\\hline
        g2 & 2168,00156018 & 2255,88720735 & 2185,99466697 \\\hline
        p1 & 7753,696801   & 7839,42728232 & 7740,21345487 \\\hline
        p2 & 6352,0804647  & 6476,84613688 & 6320,38864576 \\\hline
        t1 & 3458,82782244 & 3465,62528179 & 3457,03636612 \\\hline
        t2 & 12478,3784309 & 12235,0266878 & 12234,7514628 \\\hline
    \end{tabular}
    \captionof{table}{Abstandssummen der gemischten Instanzen der Größe 50}
\end{center}
%    \end{minipage}
%\end{figure}

%\begin{figure}[H]
%    \begin{minipage}{.45\textwidth}
\begin{center}
    \begin{tabular}{|r|S[table-format=5.8]|S[table-format=5.8]|S[table-format=5.8]|}
        \hline
        \backslashbox{Art}{Messlauf} & {\#1} & {\#2} & {\#3} \\\hline
        g1 & 6540,14581498 & 5656,20209467 & 6638,07923591 \\\hline
        g2 & 3054,30639553 & 3074,15843039 & 3139,57575107 \\\hline
        p1 & 20765,541476  & 20933,9591893 & 20762,0657001 \\\hline
        p2 & 18215,8892259 & 17769,2556008 & 18113,899511  \\\hline
        t1 & 12254,7534816 & 12363,7166301 & 12333,0075423 \\\hline
        t2 & 20995,3984479 & 21076,5563615 & 20986,0028343 \\\hline
    \end{tabular}
    \captionof{table}{Abstandssummen der gemischten Instanzen der Größe 100}
\end{center}
%    \end{minipage}\hfill%
%    \begin{minipage}{.45\textwidth}

%    \end{minipage}
%\end{figure}

\newpage

\subsection{Abstandsummen der sortierten Bilder} \label{subsec:app_sumSort}

\subsubsection{Heuristik 1} \label{subsubsec:app_heuristik1_qualität}

\begin{center}
    \begin{tabular}{|r|S[table-format=3.9]|S[table-format=3.9]|S[table-format=3.9]|S[table-format=3.10]|}
        \hline
        \backslashbox{Art}{Messlauf} & {\#1} & {\#2} & {\#3} & ø \\\hline
        g1 & 511,890729981 & 593,989925231 & 593,989925231 & 566,6235268143 \\\hline
        g2 & 378,273825497 & 249,873800568 & 297,765601491 & 308,6377425187 \\\hline
        p1 & 579,379823257 & 522,444733052 & 536,133377774 & 545,9859780277 \\\hline
        p2 & 391,556113931 & 401,822571285 & 396,699104033 & 396,6925964163 \\\hline
        t1 & 207,463887593 & 192,909859229 & 194,893376625 & 198,4223744823 \\\hline
        t2 & 838,263047345 & 869,673011642 & 838,80454995  & 848,9135363123 \\\hline
    \end{tabular}
    \captionof{table}{Heuristik 1: Abstandssummen der sortierten Instanzen der Größe 10}
\end{center}

\begin{center}
    \begin{tabular}{|r|S[table-format=5.8]|S[table-format=5.8]|S[table-format=5.8]|S[table-format=5.9]|}
        \hline
        \backslashbox{Art}{Messlauf} & {\#1} & {\#2} & {\#3} & ø \\\hline
        g1 & 2736,97257701 & 2551,29814308 & 2888,04580254 & 2725,438840877  \\\hline
        g2 & 1596,42449569 & 1486,68581038 & 1344,80513322 & 1475,971813097  \\\hline
        p1 & 6963,39812615 & 6820,63085512 & 6940,72230471 & 6908,25042866   \\\hline
        p2 & 6060,36325688 & 5988,33180827 & 6031,321592   & 6026,67221905   \\\hline
        t1 & 2834,37185044 & 3058,84209343 & 2912,4480459  & 2935,220663257  \\\hline
        t2 & 10838,6803098 & 10993,4434053 & 11082,3927782 & 10971,50549777  \\\hline
    \end{tabular}
    \captionof{table}{Heuristik 1: Abstandssummen der sortierten Instanzen der Größe 50}
\end{center}

\begin{center}
    \begin{tabular}{|r|S[table-format=5.8]|S[table-format=5.8]|S[table-format=5.8]|S[table-format=5.8]|}
        \hline
        \backslashbox{Art}{Messlauf} & {\#1} & {\#2} & {\#3} & ø \\\hline
        g1 & 4835,82352844 & 4756,40532119 & 4441,21981118 & 4677,81622027  \\\hline
        g2 & 2214,59876015 & 2293,71637533 & 2352,17062442 & 2286,82858663  \\\hline
        p1 & 20010,8195814 & 19889,8369096 & 20041,5197608 & 19980,72541727 \\\hline
        p2 & 17071,5189857 & 16979,5694001 & 16875,7776209 & 16975,62200223 \\\hline
        t1 & 10585,1207027 & 10728,5233067 & 10885,4157573 & 10733,01992223 \\\hline
        t2 & 20173,4030102 & 20210,8315004 & 19582,1992343 & 19988,8112483  \\\hline
    \end{tabular}
    \captionof{table}{Heuristik 1: Abstandssummen der sortierten Instanzen der Größe 100}
\end{center}

\subsubsection{Heuristik 2} \label{subsubsec:app_heuristik2_qualität}

\begin{center}
    \begin{tabular}{|r|S[table-format=3.9]|S[table-format=3.9]|S[table-format=3.9]|S[table-format=3.9]|}
        \hline
        \backslashbox{Art}{Messlauf} & {\#1} & {\#2} & {\#3} & ø \\\hline
        g1 & 426,752089999 & 426,752089999 & 426,752089999 & 426,752089999 \\\hline
        g2 & 249,873800568 & 249,873800568 & 249,873800568 & 249,873800568 \\\hline
        p1 & 522,444733052 & 522,444733052 & 522,444733052 & 522,444733052 \\\hline
        p2 & 373,400786545 & 373,400786545 & 373,400786545 & 373,400786545 \\\hline
        t1 & 192,909859229 & 192,909859229 & 192,909859229 & 192,909859229 \\\hline
        t2 & 838,263047345 & 838,263047345 & 838,263047345 & 838,263047345 \\\hline
    \end{tabular}
    \captionof{table}{Heuristik 2: Abstandssummen der sortierten Instanzen der Größe 10}
\end{center}

\begin{center}
    \begin{tabular}{|r|S[table-format=5.9]|S[table-format=5.9]|S[table-format=5.9]|S[table-format=5.9]|}
        \hline
        \backslashbox{Art}{Messlauf} & {\#1} & {\#2} & {\#3} & ø \\\hline
        g1 & 1715,86645274 & 1085,82959774 & 1085,82959774 & 1295,84188274 \\\hline
        g2 & 604,636533543 & 604,636533543 & 604,636533543 & 604,636533543 \\\hline
        p1 & 6587,51285829 & 6587,51285829 & 6587,51285829 & 6587,51285829 \\\hline
        p2 & 5492,5290117  & 5492,5290117  & 5492,5290117  & 5492,5290117  \\\hline
        t1 & 2798,7771451  & 2799,35407759 & 2798,7771451  & 2798,96945593 \\\hline
        t2 & 10183,381895  & 10183,381895  & 10183,381895  & 10183,381895  \\\hline
    \end{tabular}
    \captionof{table}{Heuristik 2: Abstandssummen der sortierten Instanzen der Größe 50}
\end{center}

\begin{center}
    \begin{tabular}{|r|S[table-format=5.9]|S[table-format=5.8]|S[table-format=5.8]|S[table-format=5.10]|}
        \hline
        \backslashbox{Art}{Messlauf} & {\#1} & {\#2} & {\#3} & ø \\\hline
        g1 & 1892,391478   & 1923,40968002 & 1892,391478   & 1902,730878673  \\\hline
        g2 & 990,337428662 & 1089,07985883 & 1137,06979326 & 1072,1623602507 \\\hline
        p1 & 18371,6242832 & 18371,6242832 & 18371,6242832 & 18371,6242832   \\\hline
        p2 & 14399,1155256 & 14399,1155256 & 14399,1155256 & 14399,1155256   \\\hline
        t1 & 9955,00625808 & 9955,00625808 & 9955,00625808 & 9955,00625808   \\\hline
        t2 & 18823,7799399 & 18823,7799399 & 18823,7799399 & 18823,7799399   \\\hline
    \end{tabular}
    \captionof{table}{Heuristik 2: Abstandssummen der sortierten Instanzen der Größe 100}
\end{center}

    \newpage
    \section{Laufzeiten der Heuristiken} \label{sec:app_laufzeiten}

\subsection{Heuristik 1} \label{app:heuristik1_laufzeit}

%\begin{figure}[H]
%    \begin{minipage}{.45\textwidth}
        \begin{center}
            \begin{tabular}{|r|S[table-format=1.3]|S[table-format=1.3]|S[table-format=1.3]|S[table-format=1.3]|}
                \hline
                \backslashbox{Art}{Messlauf} & {\#1} & {\#2} & {\#3} & ø \\\hline
                g1 & 0,002 & 0,002 & 0,002 & 0,002 \\\hline
                g2 & 0,002 & 0,002 & 0,002 & 0,002 \\\hline
                p1 & 0,002 & 0,002 & 0,002 & 0,002 \\\hline
                p2 & 0,002 & 0,002 & 0,002 & 0,002 \\\hline
                t1 & 0,002 & 0,002 & 0,002 & 0,002 \\\hline
                t2 & 0,002 & 0,002 & 0,002 & 0,002 \\\hline
            \end{tabular}
            \captionof{table}{Heuristik 1: Laufzeiten bei Instanzen der Größe 10}
        \end{center}
%    \end{minipage}\hfill%
%    \begin{minipage}{.45\textwidth}
        \begin{center}
            \begin{tabular}{|r|S[table-format=1.3]|S[table-format=1.3]|S[table-format=1.3]|S[table-format=1.4]|}
                \hline
                \backslashbox{Art}{Messlauf} & {\#1} & {\#2} & {\#3} & ø \\\hline
                g1 & 0,044 & 0,042 & 0,042 & 0,0427 \\\hline
                g2 & 0,042 & 0,042 & 0,042 & 0,042  \\\hline
                p1 & 0,043 & 0,042 & 0,044 & 0,043  \\\hline
                p2 & 0,043 & 0,044 & 0,042 & 0,043  \\\hline
                t1 & 0,043 & 0,043 & 0,042 & 0,0423 \\\hline
                t2 & 0,043 & 0,043 & 0,043 & 0,043  \\\hline
            \end{tabular}
            \captionof{table}{Heuristik 1: Laufzeiten bei Instanzen der Größe 50}
        \end{center}
%    \end{minipage}
%\end{figure}

%\begin{figure}[H]
%    \begin{minipage}{.45\textwidth}
        \begin{center}
            \begin{tabular}{|r|S[table-format=1.3]|S[table-format=1.3]|S[table-format=1.3]|S[table-format=1.4]|}
                \hline
                \backslashbox{Art}{Messlauf} & {\#1} & {\#2} & {\#3} & ø \\\hline
                g1 & 0,176 & 0,175 & 0,176 & 0,1757 \\\hline
                g2 & 0,175 & 0,174 & 0,174 & 0,1747 \\\hline
                p1 & 0,184 & 0,177 & 0,177 & 0,1793 \\\hline
                p2 & 0,175 & 0,180 & 0,173 & 0,176  \\\hline
                t1 & 0,175 & 0,176 & 0,175 & 0,1753 \\\hline
                t2 & 0,176 & 0,180 & 0,174 & 0,1767 \\\hline
            \end{tabular}
            \captionof{table}{Heuristik 1: Laufzeiten bei Instanzen der Größe 100}
        \end{center}
%    \end{minipage}\hfill%
%    \begin{minipage}{.45\textwidth}
%    \end{minipage}
%\end{figure}

\subsection{Heuristik 2} \label{app:heuristik2_laufzeit}

%\begin{figure}[H]
%    \begin{minipage}{.45\textwidth}
        \begin{center}
            \begin{tabular}{|r|S[table-format=1.3]|S[table-format=1.3]|S[table-format=1.3]|S[table-format=1.3]|}
                \hline
                \backslashbox{Art}{Messlauf} & {\#1} & {\#2} & {\#3} & ø \\\hline
                g1 & 0,001 & 0,001 & 0,001 & 0,001 \\\hline
                g2 & 0,001 & 0,001 & 0,001 & 0,001 \\\hline
                p1 & 0,001 & 0,001 & 0,001 & 0,001 \\\hline
                p2 & 0,001 & 0,001 & 0,001 & 0,001 \\\hline
                t1 & 0,001 & 0,001 & 0,001 & 0,001 \\\hline
                t2 & 0,001 & 0,001 & 0,001 & 0,001 \\\hline
            \end{tabular}
            \captionof{table}{Heuristik 2: Laufzeiten bei Instanzen der Größe 10}
        \end{center}
%    \end{minipage}\hfill%
%    \begin{minipage}{.45\textwidth}
        \begin{center}
            \begin{tabular}{|r|S[table-format=1.3]|S[table-format=1.3]|S[table-format=1.3]|S[table-format=1.4]|}
                \hline
                \backslashbox{Art}{Messlauf} & {\#1} & {\#2} & {\#3} & ø \\\hline
                g1 & 0,327 & 0,336 & 0,322 & 0,3283 \\\hline
                g2 & 0,638 & 0,668 & 0,661 & 0,6557 \\\hline
                p1 & 0,136 & 0,135 & 0,140 & 0,137  \\\hline
                p2 & 0,125 & 0,127 & 0,126 & 0,126  \\\hline
                t1 & 0,381 & 0,373 & 0,389 & 0,381  \\\hline
                t2 & 0,166 & 0,166 & 0,167 & 0,1667 \\\hline
            \end{tabular}
            \captionof{table}{Heuristik 2: Laufzeiten bei Instanzen der Größe 50}
        \end{center}
%    \end{minipage}
%\end{figure}

%\begin{figure}[H]
%    \begin{minipage}{.45\textwidth}
        \begin{center}
            \begin{tabular}{|r|S[table-format=2.3]|S[table-format=2.3]|S[table-format=2.3]|S[table-format=2.4]|}
                \hline
                \backslashbox{Art}{Messlauf} & {\#1} & {\#2} & {\#3} & ø \\\hline
                g1 & 10,436 & 13,532 & 10,244 & 11,404  \\\hline
                g2 & 14,771 & 14,677 & 14,596 & 14,6813 \\\hline
                p1 & 5,441  & 5,334  & 5,342  & 5,3723  \\\hline
                p2 & 22,941 & 22,983 & 22,961 & 22,9617 \\\hline
                t1 & 8,215  & 7,448  & 7,480  & 7,7143  \\\hline
                t2 & 12,111 & 12,275 & 12,146 & 12,1773 \\\hline
            \end{tabular}
            \captionof{table}{Heuristik 2: Laufzeiten bei Instanzen der Größe 100}
        \end{center}
%    \end{minipage}\hfill%
%    \begin{minipage}{.45\textwidth}
%    \end{minipage}
%\end{figure}


    \iffalse
    % Alte Daten der ursprünglichen 1. Heuristik
    \subsection{Gradienten} \label{subsec:gradienten}

    \begin{figure}[H]
        \begin{minipage}{.45\textwidth}
            \begin{center}
                \begin{tabular}{|r|r|r|r|r|}
                    \hline
                    Größe & Nr. 1 & Nr. 2 & Nr. 3 & ø \\\hline
                    10 & 0.000 & 0.000 & 0.000 & 0.000 \\\hline
                    50 & 0.001 & 0.001 & 0.001 & 0.001 \\\hline
                    100 & 0.002 & 0.002 & 0.002 & 0.002 \\\hline
                    500 & 0.007 & 0.007 & 0.007 & 0.007 \\\hline
                    1000 & 0.014 & 0.014 & 0.014 & 0.014 \\\hline
                    2500 & 0.055 & 0.054 & 0.054 & 0.0543 \\\hline
                    5000 & 0.108 & 0.110 & 0.107 & 0.1083 \\\hline
                \end{tabular}
                \captionof{table}{Laufzeit: Testinstanz g1 - Heuristik 1}
            \end{center}
        \end{minipage}\hfill%
        \begin{minipage}{.45\textwidth}
            \begin{center}
                \begin{tabular}{|r|r|r|r|r|}
                    \hline
                    Größe & Nr. 1 & Nr. 2 & Nr. 3 & ø \\\hline
                    10 & 0.000 & 0.000 & 0.000 & 0.000 \\\hline
                    50 & 0.001 & 0.001 & 0.001 & 0.001 \\\hline
                    100 & 0.002 & 0.002 & 0.002 & 0.002 \\\hline
                    500 & 0.007 & 0.007 & 0.007 & 0.007 \\\hline
                    1000 & 0.014 & 0.014 & 0.014 & 0.014 \\\hline
                    2500 & 0.053 & 0.053 & 0.053 & 0.053 \\\hline
                    5000 & 0.106 & 0.107 & 0.108 & 0.107 \\\hline
                \end{tabular}
                \captionof{table}{Laufzeit: Testinstanz g2 - Heuristik 1}
            \end{center}
        \end{minipage}
    \end{figure}

    \begin{figure}[H]
        \begin{minipage}{.45\textwidth}
            \begin{center}
                \begin{tabular}{|r|r|r|r|r|}
                    \hline
                    Größe & Nr. 1 & Nr. 2 & Nr. 3 & ø \\\hline
                    10 & 0.000 & 0.000 & 0.000 & 0.000 \\\hline
                    50 & 0.001 & 0.001 & 0.001 & 0.001 \\\hline
                    100 & 0.002 & 0.002 & 0.002 & 0.002 \\\hline
                    500 & 0.007 & 0.007 & 0.007 & 0.007 \\\hline
                    1000 & 0.014 & 0.014 & 0.014 & 0.014 \\\hline
                    2500 & 0.055 & 0.054 & 0.054 & 0.0543 \\\hline
                    5000 & 0.108 & 0.110 & 0.107 & 0.1083 \\\hline
                \end{tabular}
                \captionof{table}{Laufzeit: Testinstanz g1 - Heuristik 2}
            \end{center}
        \end{minipage}\hfill%
        \begin{minipage}{.45\textwidth}
            \begin{center}
                \begin{tabular}{|r|r|r|r|r|}
                    \hline
                    Größe & Nr. 1 & Nr. 2 & Nr. 3 & ø \\\hline
                    10 & 0.000 & 0.000 & 0.000 & 0.000 \\\hline
                    50 & 0.001 & 0.001 & 0.001 & 0.001 \\\hline
                    100 & 0.002 & 0.002 & 0.002 & 0.002 \\\hline
                    500 & 0.007 & 0.007 & 0.007 & 0.007 \\\hline
                    1000 & 0.014 & 0.014 & 0.014 & 0.014 \\\hline
                    2500 & 0.053 & 0.053 & 0.053 & 0.053 \\\hline
                    5000 & 0.106 & 0.107 & 0.108 & 0.107 \\\hline
                \end{tabular}
                \captionof{table}{Laufzeit: Testinstanz g2 - Heuristik 2}
            \end{center}
        \end{minipage}
    \end{figure}

    \subsection{Fotos} \label{subsec:fotos}

    \begin{figure}[H]
        \begin{minipage}{.45\textwidth}
            \begin{center}
                \begin{tabular}{|r|r|r|r|r|}
                    \hline
                    Größe & Nr. 1 & Nr. 2 & Nr. 3 & ø \\\hline
                    10 & 0.000 & 0.000 & 0.000 & 0.000 \\\hline
                    50 & 0.001 & 0.001 & 0.001 & 0.001 \\\hline
                    100 & 0.002 & 0.002 & 0.002 & 0.002 \\\hline
                    500 & 0.007 & 0.007 & 0.007 & 0.007 \\\hline
                    1000 & 0.014 & 0.014 & 0.014 & 0.014 \\\hline
                    2500 & 0.055 & 0.054 & 0.054 & 0.0543 \\\hline
                    5000 & 0.108 & 0.110 & 0.107 & 0.1083 \\\hline
                \end{tabular}
                \captionof{table}{Laufzeit: Testinstanz p1 - Heuristik 1}
            \end{center}
        \end{minipage}\hfill%
        \begin{minipage}{.45\textwidth}
            \begin{center}
                \begin{tabular}{|r|r|r|r|r|}
                    \hline
                    Größe & Nr. 1 & Nr. 2 & Nr. 3 & ø \\\hline
                    10 & 0.000 & 0.000 & 0.000 & 0.000 \\\hline
                    50 & 0.001 & 0.001 & 0.001 & 0.001 \\\hline
                    100 & 0.002 & 0.002 & 0.002 & 0.002 \\\hline
                    500 & 0.007 & 0.007 & 0.007 & 0.007 \\\hline
                    1000 & 0.014 & 0.014 & 0.014 & 0.014 \\\hline
                    2500 & 0.053 & 0.053 & 0.053 & 0.053 \\\hline
                    5000 & 0.106 & 0.107 & 0.108 & 0.107 \\\hline
                \end{tabular}
                \captionof{table}{Laufzeit: Testinstanz p2 - Heuristik 1}
            \end{center}
        \end{minipage}
    \end{figure}

    \begin{figure}[H]
        \begin{minipage}{.45\textwidth}
            \begin{center}
                \begin{tabular}{|r|r|r|r|r|}
                    \hline
                    Größe & Nr. 1 & Nr. 2 & Nr. 3 & ø \\\hline
                    10 & 0.000 & 0.000 & 0.000 & 0.000 \\\hline
                    50 & 0.001 & 0.001 & 0.001 & 0.001 \\\hline
                    100 & 0.002 & 0.002 & 0.002 & 0.002 \\\hline
                    500 & 0.007 & 0.007 & 0.007 & 0.007 \\\hline
                    1000 & 0.014 & 0.014 & 0.014 & 0.014 \\\hline
                    2500 & 0.055 & 0.054 & 0.054 & 0.0543 \\\hline
                    5000 & 0.108 & 0.110 & 0.107 & 0.1083 \\\hline
                \end{tabular}
                \captionof{table}{Laufzeit: Testinstanz p1 - Heuristik 2}
            \end{center}
        \end{minipage}\hfill%
        \begin{minipage}{.45\textwidth}
            \begin{center}
                \begin{tabular}{|r|r|r|r|r|}
                    \hline
                    Größe & Nr. 1 & Nr. 2 & Nr. 3 & ø \\\hline
                    10 & 0.000 & 0.000 & 0.000 & 0.000 \\\hline
                    50 & 0.001 & 0.001 & 0.001 & 0.001 \\\hline
                    100 & 0.002 & 0.002 & 0.002 & 0.002 \\\hline
                    500 & 0.007 & 0.007 & 0.007 & 0.007 \\\hline
                    1000 & 0.014 & 0.014 & 0.014 & 0.014 \\\hline
                    2500 & 0.053 & 0.053 & 0.053 & 0.053 \\\hline
                    5000 & 0.106 & 0.107 & 0.108 & 0.107 \\\hline
                \end{tabular}
                \captionof{table}{Laufzeit: Testinstanz p2 - Heuristik 2}
            \end{center}
        \end{minipage}
    \end{figure}

    \subsection{Testbilder} \label{subsec:testbilder}

    \begin{figure}[H]
        \begin{minipage}{.45\textwidth}
            \begin{center}
                \begin{tabular}{|r|r|r|r|r|}
                    \hline
                    Größe & Nr. 1 & Nr. 2 & Nr. 3 & ø \\\hline
                    10 & 0.000 & 0.000 & 0.000 & 0.000 \\\hline
                    50 & 0.001 & 0.001 & 0.001 & 0.001 \\\hline
                    100 & 0.002 & 0.002 & 0.002 & 0.002 \\\hline
                    500 & 0.007 & 0.007 & 0.007 & 0.007 \\\hline
                    1000 & 0.014 & 0.014 & 0.014 & 0.014 \\\hline
                    2500 & 0.055 & 0.054 & 0.054 & 0.0543 \\\hline
                    5000 & 0.108 & 0.110 & 0.107 & 0.1083 \\\hline
                \end{tabular}
                \captionof{table}{Laufzeit: Testinstanz t1 - Heuristik 1}
            \end{center}
        \end{minipage}\hfill%
        \begin{minipage}{.45\textwidth}
            \begin{center}
                \begin{tabular}{|r|r|r|r|r|}
                    \hline
                    Größe & Nr. 1 & Nr. 2 & Nr. 3 & ø \\\hline
                    10 & 0.000 & 0.000 & 0.000 & 0.000 \\\hline
                    50 & 0.001 & 0.001 & 0.001 & 0.001 \\\hline
                    100 & 0.002 & 0.002 & 0.002 & 0.002 \\\hline
                    500 & 0.007 & 0.007 & 0.007 & 0.007 \\\hline
                    1000 & 0.014 & 0.014 & 0.014 & 0.014 \\\hline
                    2500 & 0.053 & 0.053 & 0.053 & 0.053 \\\hline
                    5000 & 0.106 & 0.107 & 0.108 & 0.107 \\\hline
                \end{tabular}
                \captionof{table}{Laufzeit: Testinstanz t2 - Heuristik 1}
            \end{center}
        \end{minipage}
    \end{figure}

    \begin{figure}[H]
        \begin{minipage}{.45\textwidth}
            \begin{center}
                \begin{tabular}{|r|r|r|r|r|}
                    \hline
                    Größe & Nr. 1 & Nr. 2 & Nr. 3 & ø \\\hline
                    10 & 0.000 & 0.000 & 0.000 & 0.000 \\\hline
                    50 & 0.001 & 0.001 & 0.001 & 0.001 \\\hline
                    100 & 0.002 & 0.002 & 0.002 & 0.002 \\\hline
                    500 & 0.007 & 0.007 & 0.007 & 0.007 \\\hline
                    1000 & 0.014 & 0.014 & 0.014 & 0.014 \\\hline
                    2500 & 0.055 & 0.054 & 0.054 & 0.0543 \\\hline
                    5000 & 0.108 & 0.110 & 0.107 & 0.1083 \\\hline
                \end{tabular}
                \captionof{table}{Laufzeit: Testinstanz t1 - Heuristik 2}
            \end{center}
        \end{minipage}\hfill%
        \begin{minipage}{.45\textwidth}
            \begin{center}
                \begin{tabular}{|r|r|r|r|r|}
                    \hline
                    Größe & Nr. 1 & Nr. 2 & Nr. 3 & ø \\\hline
                    10 & 0.000 & 0.000 & 0.000 & 0.000 \\\hline
                    50 & 0.001 & 0.001 & 0.001 & 0.001 \\\hline
                    100 & 0.002 & 0.002 & 0.002 & 0.002 \\\hline
                    500 & 0.007 & 0.007 & 0.007 & 0.007 \\\hline
                    1000 & 0.014 & 0.014 & 0.014 & 0.014 \\\hline
                    2500 & 0.053 & 0.053 & 0.053 & 0.053 \\\hline
                    5000 & 0.106 & 0.107 & 0.108 & 0.107 \\\hline
                \end{tabular}
                \captionof{table}{Laufzeit: Testinstanz t2 - Heuristik 2}
            \end{center}
        \end{minipage}
    \end{figure}
    \section{Tabellen zu Qualitäts-Messwerten} \label{sec:app_qualität2}
    \begin{figure}[H]
        \begin{minipage}{.45\textwidth}
            \begin{center}
                \begin{tabular}{|r|r|r|r|r|}
                    \hline
                    Größe & Nr. 1 & Nr. 2 & Nr. 3 & ø \\\hline
                    10 & 0.000 & 0.000 & 0.000 & 0.000 \\\hline
                    50 & 0.001 & 0.001 & 0.001 & 0.001 \\\hline
                    100 & 0.002 & 0.002 & 0.002 & 0.002 \\\hline
                    500 & 0.007 & 0.007 & 0.007 & 0.007 \\\hline
                    1000 & 0.014 & 0.014 & 0.014 & 0.014 \\\hline
                    2500 & 0.055 & 0.054 & 0.054 & 0.0543 \\\hline
                    5000 & 0.108 & 0.110 & 0.107 & 0.1083 \\\hline
                \end{tabular}
                \captionof{table}{Qualität: Testinstanz g1 - Heuristik 1}
            \end{center}
        \end{minipage}\hfill%
        \begin{minipage}{.45\textwidth}
            \begin{center}
                \begin{tabular}{|r|r|r|r|r|}
                    \hline
                    Größe & Nr. 1 & Nr. 2 & Nr. 3 & ø \\\hline
                    10 & 0.000 & 0.000 & 0.000 & 0.000 \\\hline
                    50 & 0.001 & 0.001 & 0.001 & 0.001 \\\hline
                    100 & 0.002 & 0.002 & 0.002 & 0.002 \\\hline
                    500 & 0.007 & 0.007 & 0.007 & 0.007 \\\hline
                    1000 & 0.014 & 0.014 & 0.014 & 0.014 \\\hline
                    2500 & 0.053 & 0.053 & 0.053 & 0.053 \\\hline
                    5000 & 0.106 & 0.107 & 0.108 & 0.107 \\\hline
                \end{tabular}
                \captionof{table}{Qualität: Testinstanz g2 - Heuristik 1}
            \end{center}
        \end{minipage}
    \end{figure}
    \fi

    \newpage
    \listoftables
    \printbibliography

\end{document}