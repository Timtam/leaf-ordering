\documentclass[a4paper, 10pt, twoside, onecolumn, parskip]{scrartcl}
%\KOMAoptions{DIV=last}
\usepackage[left=4cm,right=2cm,top=2cm,bottom=4cm]{geometry}
\usepackage[ngerman]{babel}    % deutsche Einstellungen
\usepackage[utf8]{luainputenc}    % Eingabekodierung für Umlaute im Quellcode
\usepackage[T1]{fontenc}
\PassOptionsToPackage{hyphens}{url}
\usepackage[draft=false, unicode, breaklinks, ngerman, pdfdisplaydoctitle, pdfpagelayout=SinglePage%, colorlinks
]{hyperref}
\usepackage[ocgcolorlinks]{ocgx2}
\usepackage{amssymb} % mathematische Sonderzeichen
\usepackage{amsmath}
\usepackage{enumitem}
\usepackage{float}
\usepackage{slashbox}
\usepackage{comment}
\usepackage[autostyle=true,german=quotes]{csquotes}
\usepackage[backend=biber, style=alphabetic, block=ragged, backref]{biblatex}
\addbibresource{references.bib}

%\renewcommand*{\fps@figure}{htbp}

\setlist{itemsep=.5em, parsep=.0em}
%\hyphenchar\font=\string"7F

\title{Übung Algorithm Engineering} % Titel
\author{{Toni Barth} und {Max Haarbach}} % Autor
\date{\today}            % \today wird durch das aktuelle Datum ersetzt

\begin{document}
    \maketitle                % hier wird der Titel dann gedruckt
    %\tableofcontents

    \section{Heuristiken} \label{sec:heuristiken}

    \subsection{Heuristik 1: Zufallsdrehungen} \label{subsec:heuristic1}

    Die erste Heuristik führt eine bestimmte Anzahl an Drehungen, die von der Größe der Instanz abhängt, an zufällig ausgewählte Knoten aus.
    Dieser Vorgang wird wiederum je nach Größe der Instanz mehrfach durchgeführt und am Ende die Sortierung mit dem geringsten Abstand als Ergebnis ausgegeben.

    \subsection{Heuristik 2: Optimal Leaf Ordering} \label{subsec:heuristic2}

    Die zweite Heuristik nutzt das Verfahren, das Bar-Joseph und weitere für eine möglichst schnelle und optimale Sortierung von hierarchisch geclusterten Datensätzen entwickelt haben~\cite{bar2001fast}.
    Dabei wird folgender rekursiver Ansatz verfolgt:
    Sollen Kosten für einen bestimmten Knoten berechnet werden, setzen sich diese aus den Kosten der beiden Kindknoten und dem Abstand der beiden inneren Blätter dieser beiden Teilbäume.
    Sofern der Knoten, für denen Kosten berechnet werden sollen, ein Blatt bzw. einen Datensatz darstellt, betragen dessen Kosten $0$.
    Dies ist daher das Rekursionsende.
    Begonnen wird üblicherweise mit dem Wurzelknoten, da man dadurch am Ende auch die gesamten Abstandskosten berechnet hat.

    \section{Ziele} \label{sec:ziele}

    Durch die Experimente sollen sowohl
    \begin{itemize}
        \item die Laufzeiten der Heuristiken bei unterschiedlichen Größenordnungen bezüglich der Anzahl der Testobjekte als auch
        \item die Güte aufgrund der Ähnlichkeiten zu den jeweiligen Originalbildern
    \end{itemize}
    ermittelt und verglichen werden.

    \section{Faktoren} \label{sec:faktoren}

    Beim \enquote{Leaf-ordering} sind lediglich 2 Faktoren von Bedeutung:

    Zum Einen bestimmt die Größe der Bilder, die im Endeffekt die Anzahl der Testobjekte widerspiegelt, die Laufzeit der Heuristiken.
    Zum Anderen spielt auch deren Struktur oder Art eine Rolle, die sich allerdings schwer in konkrete Messgrößen oder Werte fassen lassen.

    \section{Testinstanzen} \label{sec:testinstanzen}

    Gemäß der~\nameref{sec:faktoren} werden auch die Testinstanzen, die durch Grauwert-Bilder realisiert sind, in die entsprechenden Kategorien unterteilt:
    \begin{itemize}
        \item Größen:
        \begin{itemize}
            \item 10
            \item 50
            \item 100
            \item 500
            \item 1000
            \item 2500
            \item 5000
        \end{itemize}
        \item Arten:
        \begin{itemize}
            \item (symmetrische) Testbilder
            \item Fotos der realen Welt
            \item Farb- bzw. Grau-Übergänge
        \end{itemize}
    \end{itemize}

    \section{Ergebnis-Qualität} \label{sec:qualität}

    Als Maß für die Qualität der Ergebnisse wird die Summe der Abstände aller benachbarten Blattpaare genutzt, sodass bei 10 Blättern 9 Abstände zu addieren sind.
    Die Abstände wiederum werden durch den euklidischen Abstand der entsprechenden Spaltenvektoren berechnet.
    Die Messwerte der originale Testinstanzen werden später eingefügt, da die Implementierung dazu noch fehlt.
    Für die gemischten und sortierten Instanzen sind die Werte im~\autoref{sec:app_qualität} zu finden.

    Aufgelistet werden erstmal nur Werte für die Instanzen der Größen 10, 50 und 100, da ab 500 die Laufzeit der zweiten Heuristik stark zunimmt.
    Es wurden insgesamt 3 Durchläufe durchgeführt, in denen jeweils eine Testinstanz mit bestimmter Art Größe durch die beiden Heuristiken sortiert wurde.

    %\subsection{Abstandssummen der Original-Instanzen} \label{subsec:sum_original}
    %
    %\begin{center}
    %    \begin{tabular}{|r|r|r|r|r|}
    %        \hline
    %        \backslashbox{Art}{Größe} & 10 & 50 & 100 & 500 \\\hline
    %        g1 & & & &  \\\hline
    %        g2 & & & &  \\\hline
    %        p1 & & & &  \\\hline
    %        p2 & & & &  \\\hline
    %        t1 & & & &  \\\hline
    %        t2 & & & &  \\\hline
    %    \end{tabular}
    %    \captionof{table}{Abstandssummen der originalen Testinstanzen}
    %\end{center}

    %\subsection{Abstandssummen der gemischten Bilder} \label{subsec:sum_shuffle}

    %\subsection{Abstandsummen der sortierten Bilder} \label{subsec:sum_sort}

    \section{Laufzeiten der Heuristiken} \label{sec:laufzeiten}

    Es werden je Art und Größe der Instanz 3 Messungen durchgeführt, von denen am Ende der Durchschnitt berechnet wird.
    Die Einheiten der Messungen sind jeweils Sekunden (s).
    Die Tabellen der Laufzeitenmessungen sind später im~\autoref{sec:app_laufzeiten} aufgeführt.

    \newpage
    \appendix
    \input{ergebnis-qualität.tex}
    \newpage
    \section{Laufzeiten der Heuristiken} \label{sec:app_laufzeiten}

\subsection{Heuristik 1} \label{app:heuristik1_laufzeit}

%\begin{figure}[H]
%    \begin{minipage}{.45\textwidth}
        \begin{center}
            \begin{tabular}{|r|S[table-format=1.3]|S[table-format=1.3]|S[table-format=1.3]|S[table-format=1.3]|}
                \hline
                \backslashbox{Art}{Messlauf} & {\#1} & {\#2} & {\#3} & ø \\\hline
                g1 & 0,002 & 0,002 & 0,002 & 0,002 \\\hline
                g2 & 0,002 & 0,002 & 0,002 & 0,002 \\\hline
                p1 & 0,002 & 0,002 & 0,002 & 0,002 \\\hline
                p2 & 0,002 & 0,002 & 0,002 & 0,002 \\\hline
                t1 & 0,002 & 0,002 & 0,002 & 0,002 \\\hline
                t2 & 0,002 & 0,002 & 0,002 & 0,002 \\\hline
            \end{tabular}
            \captionof{table}{Heuristik 1: Laufzeiten bei Instanzen der Größe 10}
        \end{center}
%    \end{minipage}\hfill%
%    \begin{minipage}{.45\textwidth}
        \begin{center}
            \begin{tabular}{|r|S[table-format=1.3]|S[table-format=1.3]|S[table-format=1.3]|S[table-format=1.4]|}
                \hline
                \backslashbox{Art}{Messlauf} & {\#1} & {\#2} & {\#3} & ø \\\hline
                g1 & 0,044 & 0,042 & 0,042 & 0,0427 \\\hline
                g2 & 0,042 & 0,042 & 0,042 & 0,042  \\\hline
                p1 & 0,043 & 0,042 & 0,044 & 0,043  \\\hline
                p2 & 0,043 & 0,044 & 0,042 & 0,043  \\\hline
                t1 & 0,043 & 0,043 & 0,042 & 0,0423 \\\hline
                t2 & 0,043 & 0,043 & 0,043 & 0,043  \\\hline
            \end{tabular}
            \captionof{table}{Heuristik 1: Laufzeiten bei Instanzen der Größe 50}
        \end{center}
%    \end{minipage}
%\end{figure}

%\begin{figure}[H]
%    \begin{minipage}{.45\textwidth}
        \begin{center}
            \begin{tabular}{|r|S[table-format=1.3]|S[table-format=1.3]|S[table-format=1.3]|S[table-format=1.4]|}
                \hline
                \backslashbox{Art}{Messlauf} & {\#1} & {\#2} & {\#3} & ø \\\hline
                g1 & 0,176 & 0,175 & 0,176 & 0,1757 \\\hline
                g2 & 0,175 & 0,174 & 0,174 & 0,1747 \\\hline
                p1 & 0,184 & 0,177 & 0,177 & 0,1793 \\\hline
                p2 & 0,175 & 0,180 & 0,173 & 0,176  \\\hline
                t1 & 0,175 & 0,176 & 0,175 & 0,1753 \\\hline
                t2 & 0,176 & 0,180 & 0,174 & 0,1767 \\\hline
            \end{tabular}
            \captionof{table}{Heuristik 1: Laufzeiten bei Instanzen der Größe 100}
        \end{center}
%    \end{minipage}\hfill%
%    \begin{minipage}{.45\textwidth}
%    \end{minipage}
%\end{figure}

\subsection{Heuristik 2} \label{app:heuristik2_laufzeit}

%\begin{figure}[H]
%    \begin{minipage}{.45\textwidth}
        \begin{center}
            \begin{tabular}{|r|S[table-format=1.3]|S[table-format=1.3]|S[table-format=1.3]|S[table-format=1.3]|}
                \hline
                \backslashbox{Art}{Messlauf} & {\#1} & {\#2} & {\#3} & ø \\\hline
                g1 & 0,001 & 0,001 & 0,001 & 0,001 \\\hline
                g2 & 0,001 & 0,001 & 0,001 & 0,001 \\\hline
                p1 & 0,001 & 0,001 & 0,001 & 0,001 \\\hline
                p2 & 0,001 & 0,001 & 0,001 & 0,001 \\\hline
                t1 & 0,001 & 0,001 & 0,001 & 0,001 \\\hline
                t2 & 0,001 & 0,001 & 0,001 & 0,001 \\\hline
            \end{tabular}
            \captionof{table}{Heuristik 2: Laufzeiten bei Instanzen der Größe 10}
        \end{center}
%    \end{minipage}\hfill%
%    \begin{minipage}{.45\textwidth}
        \begin{center}
            \begin{tabular}{|r|S[table-format=1.3]|S[table-format=1.3]|S[table-format=1.3]|S[table-format=1.4]|}
                \hline
                \backslashbox{Art}{Messlauf} & {\#1} & {\#2} & {\#3} & ø \\\hline
                g1 & 0,327 & 0,336 & 0,322 & 0,3283 \\\hline
                g2 & 0,638 & 0,668 & 0,661 & 0,6557 \\\hline
                p1 & 0,136 & 0,135 & 0,140 & 0,137  \\\hline
                p2 & 0,125 & 0,127 & 0,126 & 0,126  \\\hline
                t1 & 0,381 & 0,373 & 0,389 & 0,381  \\\hline
                t2 & 0,166 & 0,166 & 0,167 & 0,1667 \\\hline
            \end{tabular}
            \captionof{table}{Heuristik 2: Laufzeiten bei Instanzen der Größe 50}
        \end{center}
%    \end{minipage}
%\end{figure}

%\begin{figure}[H]
%    \begin{minipage}{.45\textwidth}
        \begin{center}
            \begin{tabular}{|r|S[table-format=2.3]|S[table-format=2.3]|S[table-format=2.3]|S[table-format=2.4]|}
                \hline
                \backslashbox{Art}{Messlauf} & {\#1} & {\#2} & {\#3} & ø \\\hline
                g1 & 10,436 & 13,532 & 10,244 & 11,404  \\\hline
                g2 & 14,771 & 14,677 & 14,596 & 14,6813 \\\hline
                p1 & 5,441  & 5,334  & 5,342  & 5,3723  \\\hline
                p2 & 22,941 & 22,983 & 22,961 & 22,9617 \\\hline
                t1 & 8,215  & 7,448  & 7,480  & 7,7143  \\\hline
                t2 & 12,111 & 12,275 & 12,146 & 12,1773 \\\hline
            \end{tabular}
            \captionof{table}{Heuristik 2: Laufzeiten bei Instanzen der Größe 100}
        \end{center}
%    \end{minipage}\hfill%
%    \begin{minipage}{.45\textwidth}
%    \end{minipage}
%\end{figure}


    \iffalse
    % Alte Daten der ursprünglichen 1. Heuristik
    \subsection{Gradienten} \label{subsec:gradienten}

    \begin{figure}[H]
        \begin{minipage}{.45\textwidth}
            \begin{center}
                \begin{tabular}{|r|r|r|r|r|}
                    \hline
                    Größe & Nr. 1 & Nr. 2 & Nr. 3 & ø \\\hline
                    10 & 0.000 & 0.000 & 0.000 & 0.000 \\\hline
                    50 & 0.001 & 0.001 & 0.001 & 0.001 \\\hline
                    100 & 0.002 & 0.002 & 0.002 & 0.002 \\\hline
                    500 & 0.007 & 0.007 & 0.007 & 0.007 \\\hline
                    1000 & 0.014 & 0.014 & 0.014 & 0.014 \\\hline
                    2500 & 0.055 & 0.054 & 0.054 & 0.0543 \\\hline
                    5000 & 0.108 & 0.110 & 0.107 & 0.1083 \\\hline
                \end{tabular}
                \captionof{table}{Laufzeit: Testinstanz g1 - Heuristik 1}
            \end{center}
        \end{minipage}\hfill%
        \begin{minipage}{.45\textwidth}
            \begin{center}
                \begin{tabular}{|r|r|r|r|r|}
                    \hline
                    Größe & Nr. 1 & Nr. 2 & Nr. 3 & ø \\\hline
                    10 & 0.000 & 0.000 & 0.000 & 0.000 \\\hline
                    50 & 0.001 & 0.001 & 0.001 & 0.001 \\\hline
                    100 & 0.002 & 0.002 & 0.002 & 0.002 \\\hline
                    500 & 0.007 & 0.007 & 0.007 & 0.007 \\\hline
                    1000 & 0.014 & 0.014 & 0.014 & 0.014 \\\hline
                    2500 & 0.053 & 0.053 & 0.053 & 0.053 \\\hline
                    5000 & 0.106 & 0.107 & 0.108 & 0.107 \\\hline
                \end{tabular}
                \captionof{table}{Laufzeit: Testinstanz g2 - Heuristik 1}
            \end{center}
        \end{minipage}
    \end{figure}

    \begin{figure}[H]
        \begin{minipage}{.45\textwidth}
            \begin{center}
                \begin{tabular}{|r|r|r|r|r|}
                    \hline
                    Größe & Nr. 1 & Nr. 2 & Nr. 3 & ø \\\hline
                    10 & 0.000 & 0.000 & 0.000 & 0.000 \\\hline
                    50 & 0.001 & 0.001 & 0.001 & 0.001 \\\hline
                    100 & 0.002 & 0.002 & 0.002 & 0.002 \\\hline
                    500 & 0.007 & 0.007 & 0.007 & 0.007 \\\hline
                    1000 & 0.014 & 0.014 & 0.014 & 0.014 \\\hline
                    2500 & 0.055 & 0.054 & 0.054 & 0.0543 \\\hline
                    5000 & 0.108 & 0.110 & 0.107 & 0.1083 \\\hline
                \end{tabular}
                \captionof{table}{Laufzeit: Testinstanz g1 - Heuristik 2}
            \end{center}
        \end{minipage}\hfill%
        \begin{minipage}{.45\textwidth}
            \begin{center}
                \begin{tabular}{|r|r|r|r|r|}
                    \hline
                    Größe & Nr. 1 & Nr. 2 & Nr. 3 & ø \\\hline
                    10 & 0.000 & 0.000 & 0.000 & 0.000 \\\hline
                    50 & 0.001 & 0.001 & 0.001 & 0.001 \\\hline
                    100 & 0.002 & 0.002 & 0.002 & 0.002 \\\hline
                    500 & 0.007 & 0.007 & 0.007 & 0.007 \\\hline
                    1000 & 0.014 & 0.014 & 0.014 & 0.014 \\\hline
                    2500 & 0.053 & 0.053 & 0.053 & 0.053 \\\hline
                    5000 & 0.106 & 0.107 & 0.108 & 0.107 \\\hline
                \end{tabular}
                \captionof{table}{Laufzeit: Testinstanz g2 - Heuristik 2}
            \end{center}
        \end{minipage}
    \end{figure}

    \subsection{Fotos} \label{subsec:fotos}

    \begin{figure}[H]
        \begin{minipage}{.45\textwidth}
            \begin{center}
                \begin{tabular}{|r|r|r|r|r|}
                    \hline
                    Größe & Nr. 1 & Nr. 2 & Nr. 3 & ø \\\hline
                    10 & 0.000 & 0.000 & 0.000 & 0.000 \\\hline
                    50 & 0.001 & 0.001 & 0.001 & 0.001 \\\hline
                    100 & 0.002 & 0.002 & 0.002 & 0.002 \\\hline
                    500 & 0.007 & 0.007 & 0.007 & 0.007 \\\hline
                    1000 & 0.014 & 0.014 & 0.014 & 0.014 \\\hline
                    2500 & 0.055 & 0.054 & 0.054 & 0.0543 \\\hline
                    5000 & 0.108 & 0.110 & 0.107 & 0.1083 \\\hline
                \end{tabular}
                \captionof{table}{Laufzeit: Testinstanz p1 - Heuristik 1}
            \end{center}
        \end{minipage}\hfill%
        \begin{minipage}{.45\textwidth}
            \begin{center}
                \begin{tabular}{|r|r|r|r|r|}
                    \hline
                    Größe & Nr. 1 & Nr. 2 & Nr. 3 & ø \\\hline
                    10 & 0.000 & 0.000 & 0.000 & 0.000 \\\hline
                    50 & 0.001 & 0.001 & 0.001 & 0.001 \\\hline
                    100 & 0.002 & 0.002 & 0.002 & 0.002 \\\hline
                    500 & 0.007 & 0.007 & 0.007 & 0.007 \\\hline
                    1000 & 0.014 & 0.014 & 0.014 & 0.014 \\\hline
                    2500 & 0.053 & 0.053 & 0.053 & 0.053 \\\hline
                    5000 & 0.106 & 0.107 & 0.108 & 0.107 \\\hline
                \end{tabular}
                \captionof{table}{Laufzeit: Testinstanz p2 - Heuristik 1}
            \end{center}
        \end{minipage}
    \end{figure}

    \begin{figure}[H]
        \begin{minipage}{.45\textwidth}
            \begin{center}
                \begin{tabular}{|r|r|r|r|r|}
                    \hline
                    Größe & Nr. 1 & Nr. 2 & Nr. 3 & ø \\\hline
                    10 & 0.000 & 0.000 & 0.000 & 0.000 \\\hline
                    50 & 0.001 & 0.001 & 0.001 & 0.001 \\\hline
                    100 & 0.002 & 0.002 & 0.002 & 0.002 \\\hline
                    500 & 0.007 & 0.007 & 0.007 & 0.007 \\\hline
                    1000 & 0.014 & 0.014 & 0.014 & 0.014 \\\hline
                    2500 & 0.055 & 0.054 & 0.054 & 0.0543 \\\hline
                    5000 & 0.108 & 0.110 & 0.107 & 0.1083 \\\hline
                \end{tabular}
                \captionof{table}{Laufzeit: Testinstanz p1 - Heuristik 2}
            \end{center}
        \end{minipage}\hfill%
        \begin{minipage}{.45\textwidth}
            \begin{center}
                \begin{tabular}{|r|r|r|r|r|}
                    \hline
                    Größe & Nr. 1 & Nr. 2 & Nr. 3 & ø \\\hline
                    10 & 0.000 & 0.000 & 0.000 & 0.000 \\\hline
                    50 & 0.001 & 0.001 & 0.001 & 0.001 \\\hline
                    100 & 0.002 & 0.002 & 0.002 & 0.002 \\\hline
                    500 & 0.007 & 0.007 & 0.007 & 0.007 \\\hline
                    1000 & 0.014 & 0.014 & 0.014 & 0.014 \\\hline
                    2500 & 0.053 & 0.053 & 0.053 & 0.053 \\\hline
                    5000 & 0.106 & 0.107 & 0.108 & 0.107 \\\hline
                \end{tabular}
                \captionof{table}{Laufzeit: Testinstanz p2 - Heuristik 2}
            \end{center}
        \end{minipage}
    \end{figure}

    \subsection{Testbilder} \label{subsec:testbilder}

    \begin{figure}[H]
        \begin{minipage}{.45\textwidth}
            \begin{center}
                \begin{tabular}{|r|r|r|r|r|}
                    \hline
                    Größe & Nr. 1 & Nr. 2 & Nr. 3 & ø \\\hline
                    10 & 0.000 & 0.000 & 0.000 & 0.000 \\\hline
                    50 & 0.001 & 0.001 & 0.001 & 0.001 \\\hline
                    100 & 0.002 & 0.002 & 0.002 & 0.002 \\\hline
                    500 & 0.007 & 0.007 & 0.007 & 0.007 \\\hline
                    1000 & 0.014 & 0.014 & 0.014 & 0.014 \\\hline
                    2500 & 0.055 & 0.054 & 0.054 & 0.0543 \\\hline
                    5000 & 0.108 & 0.110 & 0.107 & 0.1083 \\\hline
                \end{tabular}
                \captionof{table}{Laufzeit: Testinstanz t1 - Heuristik 1}
            \end{center}
        \end{minipage}\hfill%
        \begin{minipage}{.45\textwidth}
            \begin{center}
                \begin{tabular}{|r|r|r|r|r|}
                    \hline
                    Größe & Nr. 1 & Nr. 2 & Nr. 3 & ø \\\hline
                    10 & 0.000 & 0.000 & 0.000 & 0.000 \\\hline
                    50 & 0.001 & 0.001 & 0.001 & 0.001 \\\hline
                    100 & 0.002 & 0.002 & 0.002 & 0.002 \\\hline
                    500 & 0.007 & 0.007 & 0.007 & 0.007 \\\hline
                    1000 & 0.014 & 0.014 & 0.014 & 0.014 \\\hline
                    2500 & 0.053 & 0.053 & 0.053 & 0.053 \\\hline
                    5000 & 0.106 & 0.107 & 0.108 & 0.107 \\\hline
                \end{tabular}
                \captionof{table}{Laufzeit: Testinstanz t2 - Heuristik 1}
            \end{center}
        \end{minipage}
    \end{figure}

    \begin{figure}[H]
        \begin{minipage}{.45\textwidth}
            \begin{center}
                \begin{tabular}{|r|r|r|r|r|}
                    \hline
                    Größe & Nr. 1 & Nr. 2 & Nr. 3 & ø \\\hline
                    10 & 0.000 & 0.000 & 0.000 & 0.000 \\\hline
                    50 & 0.001 & 0.001 & 0.001 & 0.001 \\\hline
                    100 & 0.002 & 0.002 & 0.002 & 0.002 \\\hline
                    500 & 0.007 & 0.007 & 0.007 & 0.007 \\\hline
                    1000 & 0.014 & 0.014 & 0.014 & 0.014 \\\hline
                    2500 & 0.055 & 0.054 & 0.054 & 0.0543 \\\hline
                    5000 & 0.108 & 0.110 & 0.107 & 0.1083 \\\hline
                \end{tabular}
                \captionof{table}{Laufzeit: Testinstanz t1 - Heuristik 2}
            \end{center}
        \end{minipage}\hfill%
        \begin{minipage}{.45\textwidth}
            \begin{center}
                \begin{tabular}{|r|r|r|r|r|}
                    \hline
                    Größe & Nr. 1 & Nr. 2 & Nr. 3 & ø \\\hline
                    10 & 0.000 & 0.000 & 0.000 & 0.000 \\\hline
                    50 & 0.001 & 0.001 & 0.001 & 0.001 \\\hline
                    100 & 0.002 & 0.002 & 0.002 & 0.002 \\\hline
                    500 & 0.007 & 0.007 & 0.007 & 0.007 \\\hline
                    1000 & 0.014 & 0.014 & 0.014 & 0.014 \\\hline
                    2500 & 0.053 & 0.053 & 0.053 & 0.053 \\\hline
                    5000 & 0.106 & 0.107 & 0.108 & 0.107 \\\hline
                \end{tabular}
                \captionof{table}{Laufzeit: Testinstanz t2 - Heuristik 2}
            \end{center}
        \end{minipage}
    \end{figure}
    \section{Tabellen zu Qualitäts-Messwerten} \label{sec:app_qualität2}
    \begin{figure}[H]
        \begin{minipage}{.45\textwidth}
            \begin{center}
                \begin{tabular}{|r|r|r|r|r|}
                    \hline
                    Größe & Nr. 1 & Nr. 2 & Nr. 3 & ø \\\hline
                    10 & 0.000 & 0.000 & 0.000 & 0.000 \\\hline
                    50 & 0.001 & 0.001 & 0.001 & 0.001 \\\hline
                    100 & 0.002 & 0.002 & 0.002 & 0.002 \\\hline
                    500 & 0.007 & 0.007 & 0.007 & 0.007 \\\hline
                    1000 & 0.014 & 0.014 & 0.014 & 0.014 \\\hline
                    2500 & 0.055 & 0.054 & 0.054 & 0.0543 \\\hline
                    5000 & 0.108 & 0.110 & 0.107 & 0.1083 \\\hline
                \end{tabular}
                \captionof{table}{Qualität: Testinstanz g1 - Heuristik 1}
            \end{center}
        \end{minipage}\hfill%
        \begin{minipage}{.45\textwidth}
            \begin{center}
                \begin{tabular}{|r|r|r|r|r|}
                    \hline
                    Größe & Nr. 1 & Nr. 2 & Nr. 3 & ø \\\hline
                    10 & 0.000 & 0.000 & 0.000 & 0.000 \\\hline
                    50 & 0.001 & 0.001 & 0.001 & 0.001 \\\hline
                    100 & 0.002 & 0.002 & 0.002 & 0.002 \\\hline
                    500 & 0.007 & 0.007 & 0.007 & 0.007 \\\hline
                    1000 & 0.014 & 0.014 & 0.014 & 0.014 \\\hline
                    2500 & 0.053 & 0.053 & 0.053 & 0.053 \\\hline
                    5000 & 0.106 & 0.107 & 0.108 & 0.107 \\\hline
                \end{tabular}
                \captionof{table}{Qualität: Testinstanz g2 - Heuristik 1}
            \end{center}
        \end{minipage}
    \end{figure}
    \fi

    \newpage
    \listoftables
    \printbibliography

\end{document}